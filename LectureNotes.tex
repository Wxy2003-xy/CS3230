\documentclass{article}
\usepackage[utf8]{inputenc}    % For UTF-8 character encoding
\usepackage[T1]{fontenc}       % For proper font encoding
\usepackage{lmodern}           % Improved font rendering
\usepackage{amsmath}   % For advanced mathematical formatting
\usepackage{amssymb}   % For mathematical symbols
\usepackage{geometry}  % Adjust page margins
\usepackage{enumerate} % For custom lists
\usepackage{xcolor}  % for coloring
\usepackage{amsthm}
\usepackage{pdfpages}
\newtheorem{theorem}{Theorem}[section]
\newtheorem{lemma}[theorem]{Lemma}
\newtheorem{corollary}[theorem]{Corollary}
\newtheorem{definition}[theorem]{Definition}
\usepackage{listings}  % for code listings
\usepackage{tikz-cd}
\usepackage{forest}
\usetikzlibrary{arrows.meta}
\usetikzlibrary{arrows.meta,decorations.pathreplacing,calc}
\lstset{frame=tb,
  language=C,
  aboveskip=3mm,
  belowskip=3mm,
  showstringspaces=false,   
  columns=flexible,
  basicstyle={\small\ttfamily},
  numbers=none,
  numberstyle=\tiny\color{gray},
  keywordstyle=\color{blue},
  commentstyle=\color{brown},
  stringstyle=\color{orange},
  breaklines=true,
  breakatwhitespace=true,
  tabsize=3
}
\geometry{top=1in, bottom=1in, left=1in, right=1in}

\begin{document}

\title{Lecture 1: Asymptotic Analysis}
\author{Wang Xiyu}
\date{}
\maketitle
\section{Definition}
\begin{align*}
    &O(g): \{f: \exists c > 0 \land n_0 > 0, \forall n \geq n_0, 0 \leq f(n) \leq c \cdot g(n)\} \\
    &\Omega(g): \{f: \exists c > 0 \land n_0 > 0, \forall n \geq n_0, 0 \leq c \cdot g(n) \leq f(n)\} \\
    &\Theta(g): \{f: \exists c_1, c_2 > 0 \land n_0 > 0, \forall n \geq n_0, c_1 \cdot g(n) \leq f(n) \leq c_2 \cdot g(n)\} \\
    %
    &o(g): \{f: \forall c > 0, \exists n_0 > 0, \forall n \geq n_0, 0 \leq f(n) < c \cdot g(n)\} \\ 
    &\omega(g): \{f: \forall c > 0, \forall n \leq n_0, 0 \leq c \cdot g(n) < f(n)\} \\
\end{align*}
\section{Proof with limit}

\section{Secondary conclusions}

\[\Theta(f) = O(f) \cap \Omega(f)\]
\begin{proof}
    1. prove: $O(f) \cap \Omega(f) \subseteq \Theta(f) $\\  
    From definition: \\
    \begin{align*}
        &\Theta(f(n)): \{g: \exists c_1, c_2 > 0, \exists n_0 \geq 0, \forall n \geq n_0, c_1 \cdot f(n) \leq g(n) \leq c_2 \cdot f(n)\}\\
        &O(f(n)) \cap \Omega(f(n)): \{g: \exists c_1 > 0, \exists n_1 \geq 0, \forall n \geq n_1, 0 \leq g(n) \leq c_1 \cdot f(n) \land 
        \exists c_2 > 0, \exists n_2 \geq 0, \forall n \geq n_2, 0 \leq c_2 \cdot f(n) \leq g(n)\}
    \end{align*}
    When $0 \leq c_1 \leq c_2, O(f(n)) \cap \Omega(f(n)): 
    \{g: \exists n_1, n_2 \geq 0, \forall n \geq \{\max(n_1, n_2)\}, 0 \leq c_1 f(n) \leq g(n) \leq c_2 f(n)\}$\\
    $\implies $ $O(f) \cap \Omega(f) \subseteq \Theta(f) $\\  
    2. prove: $\Theta(f) \subseteq O(f) \cap \Omega(f)$\\ 
    $(f(n) > 0 \land, c_1 \leq, c_2) \implies (0 \leq c_1 f(n) \leq g(n) \land 0 \leq g(n) \leq c_2 f(n))$\\ 
    $\implies $  $\Theta(f) \subseteq O(f) \cap \Omega(f)$\\ 
    $\implies $  $\Theta(f) = O(f) \cap \Omega(f)$\\ 


\end{proof}
\[\lim_{n \rightarrow \infty} \frac{f(n)}{g(n)} = 0 \implies f(n) \in o(g(n))\]
\begin{proof}
    From lmit definition, $\forall \epsilon > 0, \exists n_0 \geq 0, 
    \forall n \geq n_0, \frac{f(n)}{g(n)} < \epsilon$. Let $c = \epsilon, 
    \frac{f(n)}{g(n)} < \epsilon, f(n) < c \cdot g(n)$. $f(n) \geq 0, f(n) \in o(g(n))$
\end{proof}
\[\lim_{n \rightarrow \infty} \frac{f(n)}{g(n)} = \infty \implies f(n) \in \omega(g(n))\]
\begin{proof}
    From $\lim_{n \rightarrow \infty} \frac{f(n)}{g(n)} = \infty \iff \lim_{n \rightarrow \infty} \frac{g(n)}{f(n)} = 0$, as \\
    $\lim_{x \rightarrow \infty} \frac{1}{f(x)} = \frac{1}{\lim_{x\to\infty} f(x)}, \lim_{n \rightarrow \infty} \frac{g(n)}{f(n)} = \frac{1}{\lim_{n \rightarrow \infty}\frac{f(n)}{g(n)}} = \frac{1}{\infty} = 0$.\\
    Similarly, $\forall \epsilon > 0, \exists n_0 \geq 0, 
    \forall n \geq n_0, \frac{g(n)}{f(n)} < \epsilon$. Let $c = \frac{1}{\epsilon}, 
    \frac{g(n)}{f(n)} < \epsilon, c \cdot g(n) < f(n)$.\\
\end{proof}
\[0 < \lim_{n \rightarrow \infty} \frac{f(n)}{g(n)} < \infty \implies f(n) \in \Theta(g(n))\]
\begin{proof}
    Definitions:\\
    $\Theta(g): \{f: \exists c_1, c_2 > 0 \land n_0 > 0, \forall n \geq n_0, c_1 \cdot g(n) \leq f(n) \leq c_2 \cdot g(n)\}$\\
    $O(g): \{f: \exists c > 0 \land n_0 > 0, \forall n \geq n_0, 0 \leq f(n) \leq c \cdot g(n)\}$ \\
    $o(g): \{f: \forall c > 0, \exists n_0 > 0, \forall n \geq n_0, 0 \leq f(n) < c \cdot g(n)\}$\\
    Prove: $\Theta(f) = O(f) \setminus o(f)$
    \begin{proof}
        1. $\Theta(f) \subseteq O(f) \setminus o(f)$\\
            Suppose $g(n) \in \Theta(f), \Theta(f) = O(f) \cap \Omega(f) \implies g(n) \in O(f)$. \\
            Suppose $g(n) \in o(f), (\exists c_1, c_2 > 0, \exists n_1 > 0, \forall n \geq n_1, 0 \leq c_1 \cdot f(n) \leq g(n) \leq c_2 \cdot f(n)) \land (\forall c_3 > 0, \exists n_2 > 0, \forall n \geq n_2, 0 \leq g(n) < c_3 \cdot f(n))$\\
            Let $c_3 < c_1, c_1 \cdot f(n) \leq g(n) < c_3 \cdot f(n)$, contradiction, \\
            $\implies g(n) \in \Theta(f) \rightarrow g(n) \notin o(f)$\\
            $\implies \Theta(f) \subseteq O(f) \setminus o(f)$\\
        2. \[O(f) \setminus o(f) \subseteq \Theta(f) = \]
            \[
            \begin{aligned}
            \{g:\ & (\exists c_1 > 0, \exists n_1 > 0, \forall n \geq n_1,\ 0 \leq g(n) \leq c_1 \cdot f(n)) \\
                & \land\ \neg(\forall c_2 > 0, \exists n_2 > 0, \forall n \geq n_2,\ 0 \leq g(n) < c_2 \cdot f(n)) \} \\
            \{g:\ & (\exists c_1 > 0, \exists n_1 > 0, \forall n \geq n_1,\ 0 \leq g(n) \leq c_1 \cdot f(n)) \\
                & \land\ (\exists c_2 > 0, \forall n_2 > 0, \exists n \geq n_2,\ (g(n) < 0 \lor g(n) \geq c_2 \cdot f(n))) \} \\
            \{g:\ & \exists c_1 > c_2 > 0, \forall n_2 > 0, \exists n_1 \geq n_2, \forall n \geq n_1,\ 0 \leq c_2 \cdot f(n) \leq g(n) \leq c_1 \cdot f(n) \} \\[0.5em]
            \{g:\ & \exists c_1 > c_2 > 0, \exists n_0 > 0, \forall n \geq n_0,\ 0 \leq c_2 \cdot f(n) \leq g(n) \leq c_1 \cdot f(n) \}
            \end{aligned}
            \]
    \end{proof}
\end{proof}

\section{Reflexivity}
\subsection{$f(n)\in O(f(n))$}
\begin{proof}
    Suppose $f(n) > 0$\\
    Let $c = 1, n_0 = 1, \forall n \geq n_0, 0 \leq f(n) \leq f(n), \implies f(n) \in O(f(n))$
\end{proof}
\subsection{$f(n) \in \Omega(f(n))$, $f(n) \in \Theta(f(n))$}
Similarly, by picking the right $c$ and $n_0$ it's easy to prove existential statements.

\section{Transitivity}
\subsection{$f(n) \in O(g(n)) \land g(n) \in O(h(n)) \implies f(n) \in O(h(n))$}
\begin{proof}
    \begin{align*}
        &\exists c_1 > 0, n_1 > 0, \forall n \geq n_1, 0 \leq f(n) \leq c_1 \cdot g(n)\\
        &\exists c_2 > 0, n_2 > 0, \forall n \geq n_2, 0 \leq g(n) \leq c_2 \cdot h(n)
    \end{align*}
    Let $c_1 = 1, n_0 = \max(n_1, n_2)$, $0 \leq f(n) \leq g(n) \leq c_2 \cdot h(n) \implies \exists c_2 > 0, \forall n \geq n_0, 0 \leq f(n) \leq c_2 \cdot h(n)$\\
    $\implies f(n) \in O(h(n))$  
\end{proof}
\subsection{$f(n) \in \Omega(g(n)) \land g(n) \in \Omega(h(n)) \implies f(n) \in \Omega(h(n))$}
Similar to $O(f(n))$ proof
\subsection{$f(n) \in \Theta(g(n)) \land g(n) \in \Theta(h(n)) \implies f(n) \in \Theta(h(n))$}
\begin{align*}
    &\exists c_1^{[1]}, c_2^{[1]}> 0, n_1 > 0, \forall n \geq n_1, 0 \leq c_1^{[1]} \cdot g(n) \leq f(n) \leq c_2^{[1]} \cdot g(n)\\
    &\exists c_1^{[2]}, c_2^{[2]} > 0, n_2 > 0, \forall n \geq n_2, 0 \leq c_1^{[2]} \cdot h(n) \leq g(n) \leq c_2^{[2]} \cdot h(n)
\end{align*}
Let $c_1^{[1]} = c_2^{[1]} = 1, f(n) = g(n), $\\
$0 \leq c_1^{[2]} \cdot h(n) \leq f(n) \leq c_2^{[2]} \cdot h(n) \implies f(n) \in \Theta(h(n))$
\subsection{$f(n) \in o(g(n)) \land g(n) \in o(h(n)) \implies f(n) \in o(h(n))$}
\begin{proof}
    \begin{align}
        &\forall c_1 > 0, \exists n_1 > 0, \forall n \geq n_1, 0 \leq f(n) < c_1 \cdot g(n)\\
        &\forall c_2 > 0, \exists n_2 > 0, \forall n \geq n_2, 0 \leq g(n) < c_2 \cdot h(n)
    \end{align}
From  (2), $\forall c_2 > 0, \exists n_2 > 0, \forall n \geq n_2, 0 \leq c_1 \cdot g(n) < c_1 \cdot c_2 \cdot h(n)$
    \begin{align*}
        &\forall c_1, c_2 > 0, \exists n_1, n_2 > 0, \forall n \geq \max(n_1, n_2), 0 \leq f(n) < c_1 \cdot g(n) < c_1 \cdot c_2 \cdot h(n)
    \end{align*}
    \[\forall c_1, c_2 > 0, \exists n_1, n_2 > 0, \forall n \geq \max(n_1, n_2), 0 \leq f(n) < c_1 \cdot c_2 \cdot h(n)\]
\end{proof}

\subsection{$f(n) \in \omega(g(n) \land g(n) \in \omega(h(n))) \implies f(n) \in \omega(h(n))$}
    \begin{align*}
        &\forall c_1 > 0, \exists n_1 > 0, \forall n \geq n_1, 0 \leq c_1 \cdot g(n) < f(n)\\
        &\forall c_2 > 0, \exists n_2 > 0, \forall n \geq n_2, 0 \leq c_2 \cdot h(n) < g(n)
    \end{align*}
    \[\forall c_1, c_2 > 0, \exists n_1, n_2 > 0, \geq \max(n_1, n_2), 0 \leq c_1 \cdot c_2 \cdot h(n) < c_1 \cdot g(n) < f(n)\]
\section{Symmetry: $f(n) \in \Theta(g(n)) \iff g(n)\in \Theta(f(n))$}
\begin{proof}
    \begin{align*}
        & f(n) \in \Theta(g(n)) \implies\\
        &\exists c_1, c_2> 0, n_1 > 0, \forall n \geq n_1, 0 \leq c_1 \cdot g(n) \leq f(n) \leq c_2 \cdot g(n) \implies\\
        & \left(0 \leq g(n) \leq \frac{1}{c_1}\cdot f(n) \leq \frac{c_2}{c_1}\cdot g(n)\right) \land \left( 0 \leq \frac{c_1}{c_2} \cdot g(n) \leq \frac{1}{c_2} \cdot f(n) \leq g(n)\right) \implies \\
        & 0 \leq \frac{1}{c_2} \cdot f(n) \leq g(n) \leq \frac{1}{c_1}\cdot f(n)\\
    \end{align*}
    Identical proof for the other direction
\end{proof}
\section{Complementary}
\subsection{$f(n) \in O(g(n) \iff g(n) \in \Omega(f(n)))$}
\begin{proof}
    \begin{align*}
        & f(n) \in O(g(n)) \implies \\
        & \exists c > 0 \land n_0 > 0, \forall n \geq n_0, 0 \leq f(n) \leq c \cdot g(n) \implies \\
        & 0 \leq \frac{1}{c}\cdot f(n) \leq g(n)
    \end{align*}
Proof for the other direction and the next statement are identical.
\end{proof}
\subsection{$f(n) \in o(g(n) \iff g(n) \in \omega(f(n)))$}

\section{Problems}
\subsection*{$3^{n + 1} \in O(3^n)$}
\begin{proof}
    \begin{align*}
        & 3^{n+ 1} = 3 \cdot 3^n\\
        & let c = 4, n_0 = 1, \forall n \geq n_0, \\
        & 3^{n+1} < c\cdot 3^n
    \end{align*}
\end{proof}
\subsection*{$4^n \notin O(2^n)$}
\begin{proof}
    Suppose $f(n) = 4^n \in O(2^n)$,\\
    \begin{align*}
        & \exists c > 0, n_0 > 0, \forall n \geq n_0, 0 \leq 2^{2 \times n} \leq c\cdot 2^n \implies\\
        & 0 \leq 2^n \leq c\\
        & 0 \leq n \leq \log_2c (const)\\
        & \text{n is upper bounded by a constant, a contradiction}
    \end{align*}
\end{proof}
\subsection*{$2^{\lfloor \log_2n \rfloor} \in \Theta(n)$}
\begin{proof}
    \begin{align*}
        & \text{Let } k = \lfloor \log_2 n\rfloor, 2^k \leq n < 2^{k+1} = 2 \cdot 2^k\\
        & \text{Let } c_1 = \frac{1}{2}, c_2 = 1, n_0 = 2, \\
        & \frac{n}{2} < 2^k \leq n
    \end{align*}
\end{proof}

\subsection*{For constant $i$, $a > 0$, $(n + a)^i \in O(n^i)$}
\begin{proof}
    \begin{align*}
        & \forall a > 0, \exists n_0 > a, \forall n \geq n_0, f(n) = (n + a)^i \leq (2n)^i = 2^i \cdot n^i, \\
        &\text{Let } c = 2^i, n_0 = a + 1, \forall n \geq n_0, 0 \leq f(n) \leq c \cdot n^i
    \end{align*}
\end{proof}
\subsection*{Prove: $2^{\log_2n} \in O(n)$}
\begin{proof}
    Let $c = 2, n_0 = 2, \forall n \geq n_0, 2^{\log_2n} = n \implies 0 \leq n \leq 2n$
\end{proof}
\subsection*{Prove: $2^{\log_2n} \in \Omega(n)$}
\begin{proof}
    Let $c = 0.5, n_0 = 2, \forall n \geq n_0, 2^{\log_2n} = n \implies 0 \leq 0.5n \leq n$
\end{proof}
\subsection*{Prove: $2^{\log_2n} \notin \Theta(\sqrt{n})$}
\begin{proof}
    Suppose otherwise, $\exists c_1, c_2 > 0, n_0 > 0, \forall n \geq n_0, 0 \leq c_1 \cdot \sqrt{n} \leq n \leq c_2 \sqrt{n}$,\\
    \begin{align*}
        & c_1 \leq \sqrt{n} \leq c_2\\
        & \text{However, } \lim_{n \rightarrow \infty} \sqrt{n}= \infty, \text{ thus cannot be bound by constant } c_2, \text{a contradiction.}
    \end{align*}
\end{proof}
\subsection*{Prove: $2^{\log_2n} \notin \omega(n)$}
\begin{proof}
    Suppose otherwise, $\forall c > 0, \exists n_0 > 0, \forall n \geq n_0, 0 \leq c\cdot n < n$. When $c = 2, \forall n_0 > 0, \forall n \geq n_0, 2n \geq n$, a contradiction.
\end{proof}

\subsection*{Rank order of growth}
\subsubsection*{$f_1(n) = \log n$}
By reflexivity, 
\[f_1(n) \in \Theta(\log n)\]
\subsubsection*{$f_2(n) = n!$}
By reflexivity, 
\[f_2(n) \in \Theta(n!)\]
\subsubsection*{$f_3(n) = 2^n + n$}
\[f_3(n) \in \Theta(2^n)\]
\begin{proof}
    
\end{proof}
\subsubsection*{$f_4(n) = n^{2.3} + 16n$}
\[f_4(n) \in \Theta(n^{2.3})\]
\begin{proof}
    
\end{proof}
\subsubsection*{$f_5(n) = \log(n^{2})$}
\[f_5(n) \in \Theta(\log(n))\]
\begin{proof}
    Assume base 2, $f_5(n) = \log(n^2) = 2\log(n)$, let $c_1 = 1, c_2 = 3, n_0 = 2, \\ \forall n \geq n_0, 0 \leq \log(n) \leq \log(n) \leq 3\log(n)$
\end{proof}
\[\Theta(\log(n)) \subseteq o(n^{2.3}) \subseteq o(2^n) \subseteq o(n!)\]
\begin{proof}
    \begin{align*}
        & \lim_{n \rightarrow \infty} \frac{\log(n)}{n^{2.3}} = \lim_{n \rightarrow \infty} \frac{(\log(n))'}{(n^{2.3})'} = \lim_{n \rightarrow \infty} \frac{1}{\ln(2)n^{1.3}} = 0, \\
        & \implies \forall f(n) \in \Theta(\log(n)), f(n) \in o(n)\\
    \end{align*}

    \begin{align*}
        & \lim_{n \rightarrow \infty} \frac{n^{2.3}}{2^n} = \lim_{n \rightarrow \infty} \frac{2.3\ln(n)}{n\cdot \ln(2)} = \lim_{n \rightarrow \infty} \frac{2.3}{n\cdot \ln(2)} = 0, \\
        & \implies \forall f(n) \in \Theta(n^{2.3}), f(n) \in o(2^n)
    \end{align*}

    \begin{align*}
        & \lim_{n \rightarrow \infty} \frac{2^n}{n!} = \lim_{n \rightarrow \infty} \frac{\prod_{r=1}^{n}2}{\prod_{r=1}^{n} r} = 0,  \forall n > 2\\
        & \implies \forall f(n) \in \Theta(2^n), f(n) \in o(n!)
    \end{align*}

\end{proof}



\section{Recurrence}
\subsection*{master theorem}
\begin{theorem}[Master Method]
Consider the recurrence
\[
T(n) = aT\!\left(\tfrac{n}{b}\right) + f(n),
\]
where $a, b$ are constants. Then:
\begin{itemize}
    \item[(A)] If $f(n) = O(n^{\log_b a - \varepsilon})$ for some constant $\varepsilon > 0$, then
    \[
    T(n) = O(n^{\log_b a}).
    \]

    \item[(B)] If $f(n) = \Theta(n^{\log_b a})$, then
    \[
    T(n) = \Theta(n^{\log_b a} \log n).
    \]

    \item[(C)] If $f(n) = \Omega(n^{\log_b a + \varepsilon})$ for some constant $\varepsilon > 0$, and if $f$ satisfies the smoothness condition
    \[
    af(n/b) \leq c f(n) \quad \text{for some constant } c < 1,
    \]
    then
    \[
    T(n) = \Theta(f(n)).
    \]
\end{itemize}
\end{theorem}







\subsection*{Telescoping}
For recurrence relation 
\[T: \mathbb{N} \rightarrow \mathbb{R},T(n) = a\cdot T(\frac{n}{b}) + f(n)\]
\begin{enumerate}
    \item Define operator $N_k: (N_kT)(n) = \frac{T(n)}{n^k}$, where $k = \log_ba$
    \item Transform $T(n) \rightarrow R(n) := \frac{T(n)}{n^k}$
    \item Obtain the recurrence relation of $R(n) = R(\frac{n}{b}) + \frac{f(n)}{n^k}$
    \item Define map $E_b: \mathbb{Z} \rightarrow \mathbb{N}, E_b(m) = b^m$
    \item Define operator $P_b: (P_b H)(m) = H(E_b(m)) = H(b^m)$
    \item Transform $R(n) \rightarrow S(m) := P_b(R) = R \circ E_b, S(m) = R(b^m)$
\end{enumerate} 
\[
\begin{tikzcd}[scale=1.35, transform shape]
T(n) \arrow[r, "N_k"] \arrow[d, "n \mapsto n/b"] & R(n) \arrow[r, "P_b"] \arrow[d,"n \mapsto n/b"]& S(m) \arrow[d, "m \mapsto m-1"]\\
a\cdot T(\frac{n}{b}) + f(n) \arrow[r, <->] & R(\frac{n}{b}) + \frac{f(n)}{n^k} \arrow[r, <->]& S(m - 1) + \frac{f(b^m)}{b^{m\cdot k}} \arrow[d, dashed,""]\\
\fbox{$T(n) = n^kR(n)\in \Theta(n^k H(n))$} &\arrow[l] R(n) \in \Theta(G(\log_bn)) = H(n) &\arrow[l] S(m) \in \Theta(S(0) + \sum_{i=1}^{m}\frac{f(b^i)}{b^{i\cdot k}}) = G(m)
\end{tikzcd}  
\]

\[T(n) \sim n^{\log_ba} \sum_{i=1}^{\lfloor \log_b n\rfloor}\frac{f(b^i)}{b^{i\log_b a}}\]

% GPT version: 
% \subsection*{Telescoping}
% Assume $a>0$, integer $b\ge 2$, and $n$ ranges over powers of $b$ (or replace $n/b$ by $\lfloor n/b\rfloor$). For
% \[
% T:\mathbb{N}\to\mathbb{R},\qquad T(n)=a\,T\!\left(\frac{n}{b}\right)+f(n),
% \]
% \begin{enumerate}
%   \item Define $N_k$ by $(N_k T)(n):=\dfrac{T(n)}{n^{k}}$ with $k:=\log_b a$.
%   \item Let $R(n):=(N_k T)(n)=\dfrac{T(n)}{n^k}$.
%   \item Then $R$ satisfies
%   \[
%   R(n)=R\!\left(\frac{n}{b}\right)+\frac{f(n)}{n^k}\quad\text{since }a=b^k.
%   \]
%   \item Define $E_b:\mathbb{Z}_{\ge 0}\to\mathbb{N}$ by $E_b(m)=b^m$.
%   \item Define $P_b$ (precomposition) by $(P_b H)(m):=H(E_b(m))=H(b^m)$.
%   \item Let $S(m):=(P_b R)(m)=R\circ E_b(m)=R(b^m)$.
% \end{enumerate}

% \[
% \begin{tikzcd}[scale=1.2, transform shape, column sep=large, row sep=large]
% T(n) \arrow[r, "N_k"] \arrow[d, "n \mapsto n/b"'] &
% R(n) \arrow[r, "P_b"] \arrow[d, "n \mapsto n/b"'] &
% S(m) \arrow[d, "m \mapsto m-1"'] \\
% a\,T(n/b)+f(n) \arrow[r, "="] &
% R(n/b)+\dfrac{f(n)}{n^k} \arrow[r, "="] &
% S(m-1)+\dfrac{f(b^m)}{b^{mk}}
% \end{tikzcd}
% \]

% From the rightmost recurrence,
% \[
% S(m)=S(0)+\sum_{i=1}^{m}\frac{f(b^{i})}{b^{ik}},\qquad
% R(n)=S(\log_b n),\qquad
% T(n)=n^{k}\,R(n).
% \]
\subsection*{Problems}
\subsection*{$T(n) = 4T(\frac{n}{4}) + \frac{n}{\log(n)}$}
Define $R(n) = \frac{T(n)}{n}$,
\begin{align*}
    R(n) &= \frac{4\cdot T(\frac{n}{4})}{n} + \frac{1}{\log(n)}\\
    &= R(\frac{n}{4}) + \frac{1}{\log(n)}
\end{align*}
Define $S(m) = R(4^n)$,
\begin{align*}
    S(m) & = S(m - 1) + \frac{\frac{4^m}{\log(4^m)}}{4^{m\cdot \log_4 4}}\\
    &= S(m - 1) + \frac{1}{m \log(4)}\\
    S(m) &\sim \frac{1}{\log(4)}\sum_{i=1}^{m} \frac{1}{i}\\
    S(m) &\in \Theta(\log m)\\
\end{align*}
Map back to $R(n)$ and $T(n)$, 
\begin{align*}
    S(m) \in \Theta(\log m) &\implies R(n) \in \Theta(\log(\log_4 n))\\
    &\implies T(n) \in n\log(\log(n))
\end{align*}


\subsection*{$T(n) = 4T(\frac{n}{2}) + \sqrt{n}$}
Define $R(n) = \frac{T(n)}{n^2}$,
\begin{align*}
    R(n) &= \frac{4\cdot T(\frac{n}{2})}{n^2} + \frac{\sqrt{n}}{n^2}\\
    &= R(\frac{n}{2}) + {n^{-\frac{3}{2}}}
\end{align*}
Define $S(m) = R(2^n)$,
\begin{align*}
    S(m) & = S(m - 1) + \frac{(2^m)^{-\frac{1}{2}}}{2^{m \cdot \log_2 4}}\\
    &= S(m - 1) + \frac{2^{-\frac{m}{2}}}{2^{2m}}\\
    &= S(m - 1) + \frac{1}{2^{\frac{3m}{2}}}\\
    S(m) &\sim \sum_{i=1}^{m} \frac{1}{2^{\frac{3i}{2}}} \text{ , clearly converges}\\
    S(m) &\in \Theta(1)\\
\end{align*}2
Maps back, 
\begin{align*}
    S(m) \in \Theta(1) &\implies R(n) \in \Theta(1)\\
    &\implies T(n) \in \Theta(n^2)
\end{align*}

\subsection*{$T(n) = T(\frac{n}{2}) + 2^{\sqrt{\log n}}$}
1. Using telescoping\\
$a = 0 \rightarrow \log_b a = 0, \text{there is no need for }T(n) \mapsto R(n).$\\
Define $S(m) = T(2^n)$, we have:\\
\begin{align*}
    S(m) &= S(m - 1) + 2^{\sqrt{\log{2^m}}}\\
    &= S(m - 1) + 2^{\sqrt{m}}    \\
    S(m) &\sim \sum_{i=1}^{m} 2^{\sqrt{i}} \approx \int_{1}^{m} 2^{\sqrt{x}} dx\\
\end{align*}
Solve for $ \int_{1}^{m} 2^{\sqrt{x}} dx$ by substitution: let $u = x^\frac{1}{2}$
\begin{align*}
    \frac{du}{dx} &= \frac{1}{2}x^{-\frac{1}{2}}\\
    dx &= \frac{1}{2u}\\
    \int 2^{\sqrt{x}} dx &= \int 2u\cdot 2^u du = 2\int u2^udu\\
    &= \frac{2^u(2u \ln(2) - 1)}{(\ln(2))^2} + C\\
    &= \frac{2^{\sqrt{x}}(2\sqrt{x} \ln(2) - 1)}{(\ln(2))^2} + C\\
    \int_{1}^{m} 2^{\sqrt{x}} dx &= \frac{1}{(\ln(2))^2}\left(2^{\sqrt{m}}(2\sqrt{m} \ln(2)-1)\right)
\end{align*}
Ignoring the constants, we have 
\begin{align*}
    S(m) &\sim \sum_{i=1}^{m} 2^{\sqrt{i}} \approx c\cdot 2^{\sqrt{m}}\sqrt{m}\\
    T(n) &\in \Theta(\sqrt{\log n}\cdot 2^{\sqrt{\log n}})
\end{align*}
\subsection*{Recurrence relation of merging $k$ arrays of size $n$}
\[T(k, n) = 2T(\frac{k}{2}, n) + \frac{nk}{2}\]
\[T(\frac{k}{2}, n) = 2T(\frac{k}{4}, n) + \frac{nk}{4}\]
Observed that the non-recursive funcion can be tightly bounded by $nk$,
\[T(k, n) = 2T(\frac{k}{2}, n) + cnk\]


With respect to $k$, Define $R(k, n) = \frac{T(k, n)}{k}$
\begin{align*}
    R(k, n) &= \frac{2 T(\frac{k}{2}, n)}{k} + cn\\
    &= R(\frac{k}{2}, n) + cn\\
\end{align*}
Define $S(m, n) = R(2^m, n)$,\\
$f(k, n) = cnk \rightarrow f(2^m, n) = cn2^m$, 
\begin{align*}
    S(m, n) &= S(m - 1, n) + \frac{cn2^m}{2^m}\\
    & \sim cn\sum_{i=1}^{m} 1\\
    S(m, n) &\in \Theta(nm)
\end{align*}
Maps back to $R(n)$,
\begin{align*}
    S(m, n) \in \Theta(n) &\implies R(k, n) \in \Theta(n\log(k))\\
    &\implies T(k, n) \in \Theta(nk\log(k))
\end{align*}




\section{Proof of Correctness}
\subsection*{Iterative algorithms}
Steps
\begin{enumerate}
    \item Loop invariant
    \item Initialization
    \item Maintenance
    \item Termination
\end{enumerate}
\begin{lstlisting}[mathescape=true]
    Dijkstra($G(V, E), s \in V$):
        $d(s) = 0$
        $R = \{s\}$
        while($R \notin V$):
            $v \in V \setminus R, v = \min_{u\in R}(d(u) + w(u, v))$
            $d(v) = \min_{u \in R}(d(u) + w(u, v))$
            $R \leftarrow v$
\end{lstlisting}
Loop invariant: 
\[\forall u \in R, d(u) = dist(s, u)\]
Initialization:
\end{document}
