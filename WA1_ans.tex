\documentclass{article}
\usepackage[utf8]{inputenc}    % For UTF-8 character encoding
\usepackage[T1]{fontenc}       % For proper font encoding
\usepackage{lmodern}           % Improved font rendering
\usepackage{amsmath}   % For advanced mathematical formatting
\usepackage{amssymb}   % For mathematical symbols
\usepackage{geometry}  % Adjust page margins
\usepackage{enumerate} % For custom lists
\usepackage{xcolor}  % for coloring
\usepackage{amsthm}
\usepackage{pdfpages}
\newtheorem{theorem}{Theorem}[section]
\newtheorem{lemma}[theorem]{Lemma}
\newtheorem{corollary}[theorem]{Corollary}
\newtheorem{definition}[theorem]{Definition}
\usepackage{listings}  % for code listings
\usepackage{tikz-cd}
\usepackage{forest}

\usetikzlibrary{arrows.meta}
\usetikzlibrary{arrows.meta,decorations.pathreplacing,calc}
\lstset{frame=tb,
  language=C,
  aboveskip=3mm,
  belowskip=3mm,
  showstringspaces=false,   
  columns=flexible,
  basicstyle={\small\ttfamily},
  numbers=none,
  numberstyle=\tiny\color{gray},
  keywordstyle=\color{blue},
  commentstyle=\color{brown},
  stringstyle=\color{orange},
  breaklines=true,
  breakatwhitespace=true,
  tabsize=3
}
\geometry{top=1in, bottom=1in, left=1in, right=1in}

\begin{document}

\title{WA 1}
\author{Wang Xiyu}
\date{}
\maketitle

\section{Asymptotic Analysis}
\subsection{$f(n) = 3n + 1,\; g(n) = 4n$}
$f(n) \in \Theta(g(n))$
\begin{proof}
    $\Theta(g): \{f: \exists c_1, c_2 > 0 \land n_0 > 0, \forall n \geq n_0, c_1 \cdot g(n) \leq f(n) \leq c_2 \cdot g(n)\}$ \\

    Let $c_1 = \frac{3}{4}, c_2 = 1, n_0 = 1$, trivial
\end{proof}
\subsection{$f(n) = 3^n,\; g(n) = 2^n \cdot n^{10000}$}
$f(n) \in \omega(g(n))$
\begin{proof}
    $\lim_{n \rightarrow \infty} \frac{f(n)}{g(n)} = \lim_{n \rightarrow \infty}
     \frac{3^n}{2^n \cdot n^{10000}} = \lim_{n \rightarrow \infty} \frac{1.5^n}{n^{10000}} 
     = \lim_{n \rightarrow \infty} \frac{\ln(1.5)1.5^n}{10000n^{9999}}$\\
     $= ... = \lim_{n \rightarrow \infty} \frac{(\ln 1.5)^{10000}1.5^n}{10000!} = \infty \implies f(n) \in \omega(g(n))$
\end{proof}
\subsection{$f(n) = 2^n,\; g(n) = 2^{n+1}$}
$f(n) \in \Theta(g(n))$
\begin{proof}
    $g(n) = 2\cdot 2^n$, let $c_1 = \frac{1}{2}, c_2 = 1, n_0 = 2$, \\
    $\forall n \geq n_0, 1 \cdot 2^n \leq 2^n \leq 2\cdot 2^n \implies f(n) \in \Theta(g(n))$
\end{proof}
\subsection{$f(n) = 2^{(2^n)},\; g(n) = 2^{(2^{n+1})}$}
$f(n) \in o(g(n))$
\begin{proof}
    Let $u = 2^n$, $f(n) = 2^u, g(n) = 2^{2\cdot u}, g(n) = (f(n))^2. \forall n \in \mathbb{Z}^+, f(n) > 1$,\\
    $f(n) < (f(n))^2, 0 < f(n) < g(n) \implies f(n) \in o(g(n))$
\end{proof}
\subsection{$f(n) = \sum_{i=1}^n \frac{1}{2^i},\; g(n) = \sum_{i=1}^n \frac{i^{10000}}{2^i}$}
$f(n) \in \Theta(g(n))$
\begin{proof}
   $f(n) = \frac{\frac{1}{2}(\frac{1}{2^n} -1)}{\frac{1}{2} - 1} = 1 - \frac{1}{2^n}$, 
   trivial to show this is tightly upper bounded by constant. $f(n) \in \Theta(1)$\\
   Next we will show that the summation $g(n)$ converges. \\
   \[\lim_{n \rightarrow \infty} \frac{\frac{n^{10000}}{2^n}}{\frac{(n-1)^{10000}}{2^{n-1}}} 
   = \lim_{n \rightarrow \infty} \frac{n^{10000}}{2(n-1)^{10000}} = \lim_{n \rightarrow \infty} \frac{10000 n^{9999}}{20000(n-1)^{9999}} = \dots\]
   \[= \lim_{n \rightarrow \infty} \frac{10000!}{2(10000!)} = \frac{1}{2} < 1 \implies \sum_{i=1}^n \frac{i^{10000}}{2^i} \quad \text{converges}\]
   $g(n) \in \Theta(1) \implies h(n) \in \Theta(g(n)) \text{ by symmetry }\implies f(n) \in \Theta(g(n))$ by transitivity.
\end{proof}

\subsection{$f(n) = \log\log (n^n),\; g(n) = \log\log(2^{\sqrt{n}})$}
$f(n) \in \Theta(g(n))$
\begin{proof}
    $f(n) = \log(n \log(n )) = \log(n) + \log(\log(n))$\\
    $g(n) = \log(\sqrt{n}\log 2) = \log(\sqrt{n}) + \log(\log2) = \frac{1}{2}\log(n) + \log\log(2)$\\
    $L = \frac{f(n)}{g(n)} = \frac{\log n + \log(\log n)}{0.5 \log n + \log (\log 2)} = 
    \frac{1 + \frac{\log\log(n)}{\log(n)}}{\frac{1}{2} \frac{\log\log(2)}{\log(n)}}$\\
    $\lim_{n\rightarrow \infty} L = 2$\\
    $0 < \lim_{n \rightarrow \infty} \frac{f(n)}{g(n)} < \infty \implies f(n) \in \Theta(g(n))$
\end{proof}
\subsection{$f(n) = 2^{\sqrt{\log n}},\; g(n) = \sqrt{n}$}
$f(n) \in o(g(n))$
\begin{proof}
    $L = \frac{f(n)}{g(n)}, \log(L) = \log(f(n)) - \log(g(n)) = \sqrt{\log n}\log(2) - \frac{1}{2} \log(n) $\\
    $= -\frac{1}{2}\log n (1 - \frac{\log 2 \sqrt{\log n}}{\log n}) = -\frac{1}{2}\log n (1 - \frac{\log 2 }{\sqrt{\log n}}) $\\
    $\lim_{n\rightarrow\infty} \log(L) = -\infty \implies \lim_{n\rightarrow\infty} L = e^{-\infty} = 0^+ \implies f(n) \in o(g(n))$ 
\end{proof}

\subsection{$f(n) = 2^{\sqrt{\log n}},\; g(n) = 2^{2^{\sqrt{\log\log n}}}$}
$f(n) \in \omega(g(n))$
\begin{proof}
Let $u = \log\log n$, $f(u) = 2^{\sqrt{2^u}}, g(u) = 2^{\sqrt{u}}$\\
$L = \frac{f(u)}{g(u)} = \frac{2^{\sqrt{2^u}}}{2^{\sqrt{u}}}, \log(L) = \log(2) (\sqrt{2^u} - \sqrt{u}), \lim_{u \rightarrow \infty}\log(L) = \infty \implies
\lim_{u \rightarrow \infty}L = \infty$\\
$u = \log\log n $ is monotonous and unbounded, $\lim \frac{f(u)}{g(u)} = \infty \implies \lim \frac{f(n)}{g(n)} = \infty \implies f(n) \in \omega(g(n)) $
\end{proof}


\subsection{$f(n) = n^{\frac{1}{\log n}},\; g(n) = n^{\frac{1}{(\log n)^2}}$}
$f(n) \in \Theta(g(n))$
\begin{proof}
    Let $L = \frac{f(n)}{g(n)} = n^{\frac{1}{\log n} - \frac{1}{(\log n)^2}} = n^{\frac{\log n - 1}{(\log n)^2}}$\\
    $\log(L) = \log n (\frac{\log n - 1}{(\log n)^2}) = \frac{\log n - 1}{\log n} = 1 - \frac{1}{\log n}, $\\
    $\lim_{n \rightarrow \infty} \log(L) = 1 \implies \lim_{n \rightarrow \infty} L = 2 \implies f(n) \in \Theta(g(n))$
\end{proof}

\subsection{$f(n) = n^{\frac{1}{\log n}},\; g(n) = n^{\frac{1}{\sqrt{\log n}}}$}
$f(n) \in o(g(n))$
\begin{proof}
    Let $L = \frac{f(n)}{g(n)} = n^{\frac{1}{\log n} - \frac{1}{(\log n)^\frac{1}{2}}} = n^{\frac{1-(\log n)^{\frac{1}{2}}}{\log n}}$\\
    $\log(L) = \log n (\frac{1-\sqrt{\log n}}{\log n}) = 1-\sqrt{\log n} $\\
    $\lim_{n \rightarrow \infty} \log(L) = -\infty \implies \lim_{n \rightarrow \infty} L = 0^+ \implies f(n) \in o(g(n))$
\end{proof}

\section{Recurrence}

\subsection{$T(n) = 16T(\frac{n}{4}) + 32n\log^{128}n$}
Define $R(n) = \frac{T(n)}{n^2}$:
\begin{align*}
    R(n) &= \frac{T(\frac{n}{4})}{(\frac{n}{4})^2} + \frac{32\log^{128}n}{n}\\
    &= R(\frac{n}{4}) + \frac{32\log^{128}n}{n}
\end{align*}
Define $S(m) = R(4^m)$:
\begin{align*}
    S(m) &= S(m - 1) + \frac{32 \cdot 4^m (\log(4^m))^{128}}{4^{m \cdot \log_4(16)}}\\
    &= S(m -1) + \frac{32\cdot 4^m m^{128}}{16^m}\\
    &= S(m-1) + \frac{32(m)^{128}}{4^m}\\
    S(m) &\in \Theta(S(0) + 32\sum_{i=1}^{m} \frac{i^{128}}{4^i})
\end{align*}
It's easy to show that $f(n) = \sum_{i=0}^{m}\frac{i^{128}}{4^i} \in O(1)$ as 
\[\lim_{n\rightarrow \infty} \frac{n^{128}}{4^n} = \lim_{n\rightarrow \infty} \frac{128 n^{127}}{\ln(4) 4^n} 
= \dots = \lim_{n \rightarrow \infty} \frac{c_1 }{c_2 4^n} = 0\quad \text{for some constant } c_1, c_2\]
$\lim_{n \rightarrow \infty} \frac{\frac{n^{128}}{4^n}}{1} = 0 \implies f(n) = \frac{n^{128}}{4^n} \in o(1) \implies f(n) \in O(1)$

\begin{align*}
    S(m) &\in O(1)\\
    R(n) &\in O(1)\\
    T(n) &\in O(n^2)
\end{align*}
\subsection{$T(n) = 16T(\frac{n}{4}) + 64n^2 \log^8 n + 32n \log^{128} n$}
Define $R(n) = \frac{T(n)}{n^2}$:
\begin{align*}
    R(n) &= \frac{16\cdot T(\frac{n}{4})}{(\frac{n^2}{4})} + \frac{64n^2(\log(n))^8}{n^2} + \frac{32n\log^{128}n}{n^2}\\
    R(n) &= \frac{16\cdot T(\frac{n}{4})}{(\frac{n^2}{4})} + 64(\log(n))^8 + \frac{32\log^{128}n}{n}\\
\end{align*}
Define $m = 4^n$, $n = \log(m)$, $S(m) = R(4^m)$
\begin{align*}
    S(m) &= S(m - 1) + \frac{64(4^m)^2(\log(4^m))^8}{4^{m\cdot \log_4 16}}+ \frac{32 (4^m) (\log(4^m))^{128}}{4^{m\cdot \log_4 16}}\\
    &= S(m - 1) + 64(m\log 4)^8 + \frac{32(m(\log 4))^{128}}{4^m}\\
    S(m) &\in O\left(S(0) + 64\cdot (\log 4)^8\sum_{i=1}^{m}i^8 + 32\cdot (\log 4)^{128}\sum_{i=1}^{m} \frac{i^{128}}{4^i}\right) 
\end{align*}
We have shown in 1.1 that $g(m) = \sum_{i=1}^{m} \frac{i^{128}}{4^i} \in O(1)$.\\
    \[\sum_{i=1}^{n} i^p = \frac{1}{p+1}\sum_{j=0}^{p} (-1)^j \binom{p+1}{j}\, B_j \, n^{p+1-j},
    \quad p\in\mathbb{N}\] 
    where \(B_j\) are the Bernoulli numbers.    \text{(https://en.wikipedia.org/wiki/Faulhaber)}
    \[h(k) = \sum_{i=1}^{k}i^8 \in O(k^9)\]
    \begin{align*}
        S(m) &\in O(S(0) + m^9)\\
        &\in O(m^8)\\
        R(n) &\in O((\log_4 n)^9)\\
        T(n) &\in O(n^2 \log^9n)
    \end{align*}


\subsection{$T(n) = T(\sqrt{n}) + \log(n)$}
Define $n = 2^m$, $m = \log(n)$ 
\begin{align*}
    T(n) &= T(n^{\frac{1}{2}}) + \log(n)\\
    T(2^m) &= T(2^{\frac{1}{2} m}) + \log(2^m)\\
    &= T(2^{\frac{1}{2} m}) + m\\
    R(m) &= R(\frac{m}{2}) + m
\end{align*}
Define $m = 2^k$, $k = \log(m)$
\begin{align*}
    R(2^k) &= R(\frac{2^k}{2}) + 2^k\\
    S(k) &= S(k-1) + 2^k\\
    S(k) &\in O(S(0) + \sum_{i=1}^{k} 2^i) = O(1 + 2^{k + 1} + 2) = O(2^k)
\end{align*}
Map back to $R(m)$ and $T(n)$:
\begin{align*}
    S(k) &\in O(2^k)\\
    \implies R(m) &\in O(2^{\log(m)}) = O(m)\\
    \implies T(n) &\in O(\log(n))
\end{align*}

\subsection{$T(n) = T(\sqrt{n}) + \log\log(n)$}
Define $n = 2^m$, $m = \log(n)$ 
\begin{align*}
    T(n) &= T(n^{\frac{1}{2}}) + \log\log(n)\\
    T(2^m) &= T(2^{\frac{1}{2} m}) + \log\log(2^m)\\
    &= T(2^{\frac{1}{2} m}) + \log(m)\\
    R(m) &= R(\frac{m}{2}) + \log(m)
\end{align*}
Define $m = 2^k$, $k = \log(m)$
\begin{align*}
    R(2^k) &= R(\frac{2^k}{2}) + \log(2^k)\\
    S(k) &= S(k-1) + k\\
    S(k) &\in O(S(0) + \sum_{i=1}^{k} i) = O(\frac{k(k + 1)}{2}) = O(k^2)
\end{align*}
Map back to $R(m)$ and $T(n)$:
\begin{align*}
    S(k) &\in O(k^2)\\
    \implies R(m) &\in O(\log^2(m))\\
    \implies T(n) &\in O((\log\log(n))^2)
\end{align*}



\end{document}